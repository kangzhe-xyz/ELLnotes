\documentclass[../main.tex]{subfiles}

\begin{document}

\chapter{Language and Age}

\section{Making Sense of Age}
\textbf{Age Descriptors} \begin{itemize}
	\item generic labels: young, old
	\item age group reference: infant, toddler, child, teen, adult, senior
	\item generational reference: baby boomers, generation x, millennials, generation z
\end{itemize}

\subsection{Analysing Age-related Language}
\begin{itemize}
	\item an insight to the values of the society (eg. the use of honorific terms when addressing those who are older),
	\item a representation of our understanding of individuals of different age groups and their general condition (eg. motherese: the kind of language mothers (or not mothers) tend to use when talking to young babies),
	\item to account for language change and the source of certain attitudes towards language use (the young at the frontline of language innovation, those who are older want to hold on to what they find more familiar). 
\end{itemize}

\subsection{Ways of Looking at Language and Age}
\begin{itemize}
	\item Language targeting an age group
	\item Language associated with an age group
	\item Lexis that is central to that age group
	\item Voice quality
\end{itemize}

\section{Ways of Describing Individuals of Different Ages}
\begin{itemize}
	\item young: bratty, sprightly, full of potential, cute
	\item not so young: senior, silver generation, coot, cute
\end{itemize}

Interestingly the word ``cute'' can describe these two different ages, though their motivations/attitudes are different.

\section{Directed Languages}
There are child-directed and elder-directed languages. There are some similarities to consider:
\begin{itemize}
	\item Higher pitch,
	\item slower speed,
	\item more pauses,
	\item clearer pronunciation, and
	\item exaggerated intonation.
\end{itemize}

\subsection{Significance of an Ageing Population}
\begin{itemize}
	\item More provisions and considerations for them
	\item A re-imagination of what it means to be an elderly person \begin{itemize}
		\item euphemisms to give positive connotations and associations with these groups. \begin{itemize}
			\item \textit{Pioneer Generation}: connotations of being the ones responsible for Singapore's beginnings, distinguished group of people as they are the only batch that can be called this.
			\item \textit{Merdeka Generation}: the ones who were around during the independence of Singapore.
	\end{itemize}
		\item For ease of administration, just give names to these groups instead of ``for those born between 19xx--19yy''.
		\item Also not age-centric or age-focused, but highlights the achievements and milestones of Singapore's history.
	\end{itemize}
	\item Why do such changes have to be made?
	\item How are these changes manifested using language?
\end{itemize}

\end{document}
