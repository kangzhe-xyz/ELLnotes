\documentclass[../main.tex]{subfiles}

\begin{document}

\section{Language and Age}

\subsection{Making Sense of Age}
\textbf{Age Descriptors} \begin{itemize}
	\item generic labels: young, old
	\item age group reference: infant, toddler, child, teen, adult, senior
	\item generational reference: baby boomers, generation x, millennials, generation z
\end{itemize}

\subsubsection{Analysing Age-related Language}
\begin{itemize}
	\item an insight to the values of the society (eg. the use of honorifics when addressing those who are older),
	\item a representation of our understanding of indivituals of different age groups and their general condition (eg. motherese: the kind of language mothers (or not mothers) tend to use when talking to young babies),
	\item to account for language change and the source of certain attitudes towards language use (the young at the frontline of language innovation, those who are older want to hold on to what they find more familiar). 
\end{itemize}

\subsubsection{Ways of Looking at Language and Age}
\begin{itemize}
	\item Language targeting an age group
	\item Language associated with an age group
	\item Lexis that is central to that age group
	\item Voice quality
\end{itemize}

\subsection{Ways of Describing Individuals of Different Ages}
\begin{itemize}
	\item young: bratty, sprightly, full of potential, cute
	\item not so young: senior, silver generation, coot, cute
\end{itemize}

\end{document}
