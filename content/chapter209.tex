\documentclass[../main.tex]{subfiles}

\begin{document}
    \section{Language and Culture}
    \subsection{Sapir-Whorf Hypothesis}
        {\it ``Human beings do not live in the objective world alone, nor in the world of social activity as ordinarily understood, but are very much at the mercy of the particular language which has become the medium of expression for their society... we see and hear and otherwise experience very largely as we do because the language habits of our community predispose certain choices of interpretation.''} \\ \hfill ---Edward Sapir, 1929 \par 
        {\it ``The background linguistic system (in other words, the grammar) of each language is not merely the reproducing instrument for voicing ideas but rather is itself the shaper of ideas, the programme and guide ''} \\ \hfill ---Benjamin Whorf, 19 \par 
        
        \subsection{Lingustic Determinism}
            \edfn{Linguistic Determinism}{In this version of the Sapir-Whorf hypothesis, it is said that language determines how we perceive and think about the world.}
            Under no circumstance should linguistic determinism be accepted as a theory due to its indefensible and extreme nature.
            \exmp{The Hopi Indians}{It was observed that they do not have grammatical distinction for tenses and hence was hypothesised by some based on the basis of lingusitic determinism that their perception of time is different from speakers of other languages. It is obvious to see how this is an absurd proposition, as time can be described without the need of tenses.}
        
        \subsection{Lingustic Relativism}
            This is the more accepted theory in modern sociolinguistics. 
            \edfn{Linguistic Relativism}{This is a more moderated version of the Sapir-Whorf hypothesis. It proposes that different languages encode different categories, and speakers of different langauges think of the world in different ways.}
            \exmp{Colours}{The perception of colours vary across languages: \begin{itemize}
                \item Blue and green are one word in Navaho,
                \item Russian has two distinct words for dark and light blue (\textit{siniy} and \textit{goluboy}),
                \item Zuni doesn't distinguish between orange and yellow.
            \end{itemize}}
            \exmp{Locations}{English and Italian \begin{itemize}
                \item English: ride `on' a bicycle and go `on' a country.
                \item Italian: ride `in' a bicycle and go `in' a country,
            \end{itemize}
            English and Japanese
            \begin{itemize}
                \item English: ring is placed `on' a finger, a finger is placed `in' a ring,
                \item Japanese: for both cases, the concept of \textit{kitta} is that the ring gets `integrated' with the wearer.
            \end{itemize}}

        \subsection{Criticism}
            One main reason that linguistic determinism is not accepted is because people's thoughts and perceptions are not determined by the words and structures of their language; it is possible to learn the thinking and experiences of other cultures without needing the words explicitly in the lexicon of another language that is not said culture. For the Hopi Indian study, the Hopi Indians did not utilise tenses but had other measures in place as a means of relating to time such as making references to concepts such as days, weeks, months, or simply having a calendar in place. \par
            \exmp{The Munduruku People}{The Munduruku people in the Amazon have no words for triangle, square, or rectangle. Yet, their children are still able to distinguish shapes and understand geometric concepts as well as those who have the vocabulary to account for such features.}
            \exmp{The Great Dani Valley People}{The Great Dani Valley in New Guinea have only two words to describe colour: light and dark. At the height of lingusitic determinism, some linguists proposed that the Great Dani Valley people had monochromatic vision due to the lack of words available to describe different hues. Yet, they are still able to distinguish and recognise most colours on the spectrum; perception of colour is dependent on one's retina---not one's tongue.}
            Perhaps the following was the most contentious piece of evidence that disproves determinism:
            \exmp{Inuits}{Antrhopologists have determined that the Inuits have no more words than English for snow\footnote{for every type of every variant of snow one could encounter, there were more words in the English language to describe those than in the language of the Inuits.}. They are perhaps more adept at using precise terminology to describe the weather conditions like how a doctor would be able to diagnose ailments with more precision than a layperson.}

    
\end{document}