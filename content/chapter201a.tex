\documentclass[../main.tex]{subfiles}

\begin{document}
    \section{Fundamentals in Sociolingustics}
    \subsection{Diglossia}
    \edfn{Diglossia}{Two distinct varieties of a language existing together. eg.: High and Low variety of language.}
    \exmp{Matched Guise Studies}{\begin{itemize}
        \item Play audio recordings of someone speaking.
        \item Then asked some questions.
        \item The speakers were chosen based on criteria such as gender, accent, etc.
        \item It was found that respondents tended to try to match their accents with the interviewers.
        \item Criticism: \begin{itemize}
            \item May only represent one accent based on the interviewers
            \item Context not really presented
            \item ``I am going to record you''---you are being observed; some measurable level of consciousness of self.
        \end{itemize}
    \end{itemize}}

    \subsection{Accommodation Theory}
    \edfn{Accommodation Theory}{Proposed by Howard Giles in 1990, it is found that speakers will converge to reduce social distance and diverge to increase social distance between other interlocutors in communication.}
    Language can be used to include or exclude people.

    \subsection{Multistyle}
    \edfn{Multistyle}{(Le Page, Tabounet-Keller, 1985) Individial creates for himself or herself the patterns of his linguistic behaviour so as to resemble those of a group or groups with which he wishes to be distinguished.}

    \subsection{Code Switching}
    Code switching is changing language mid-utterance, while style switching is switching between varieties. However because varieties and language may not be easily discernible in some contexts, both terms may be used interchangeably unless beyond a reasonable doubt. Code mixing is the mix of two or more languages in discourse. 

    \subsubsection{Extensions to Code Switching Theory}
    \begin{itemize}
        \item `we' vs `they' codes (Gumperz, 1982).
        \item Strategic ambiguity (Heller, 1992): people will code-switch to gain advantage.
        \item Myer-Scottons (1989) presented Nigerian use of code-mixing and code-switching being linked to familiarity.
    \end{itemize}

    \subsection{Marked and Unmarked Choices}
    Proposed by Myer-Scottons (1989), unmarked choices are choices of code which conform to community norms and participants' expectations. Marked choices are a departure from the expected choice, and thus carries a ``shock value'', hinting at an ulterior motive of the speaker. The general effect is to negotiate a change in the expected social distance between the participants, either increasing or decreasing it. 
\end{document}