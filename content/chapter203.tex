\documentclass[../main.tex]{subfiles}

\begin{document}
    \chapter{Language Change: Pidgins and Creoles}

    \section{Pidgins}
    \edfn{Pidgin}{A pidgin is created when two speakers with no common language have to communicate with each other.}
    Pidgins have been described as `incomplete' languages. \par
    They are formed when speakers of two different languages come into contact and have a need to communicate, but for limited purposes such as trade. \par

    \subsection{Colonisation}
    Colonisation is an example of where pidgins can arise. The three Gs of colonisation are God, Greed, Glory. There are three types of colonisation: 
    \begin{itemize}
        \item \textbf{Displacement:} substantial settlement (eg. North America)
        \item \textbf{Subjection:} sparse settlement (eg. Nigeria)
        \item \textbf{Replacement:} new labour from elsewhere (eg. Jamaica)
    \end{itemize}
    Most pidgins are based on European languages and this is a reflection of their history of colonisation.

    \exmp{West Africa}{
    In the 19th century, the whole of Africa was shared out among European powers. Large black population of Africa was administered by a small group of British Officials. \par 
    Africa was supplying raw materials for the UK. The officials needed to communicate with the native population. As a result, Guinea Coast Creole English was the lingua franca in that region.
    }

    Some common features of English-based pidgins:
    \begin{itemize}
        \item distinguishable from dialects as they are clearly made up from two different language sources.
        \item simplified grammatical structures, primarily because it was for limited use.
        \item vocabulary dominated by English words.
        \item small vocabulary (700--2000 words).
        \item small range of linguistic functions.
    \end{itemize}

    \subsection{Structures of Pidgins}
    \edfn{Strate}{A strate is a language that influences, or is influenced by another language due to contact.}
    A pidgin contains a substrate and a superstrate. If the pidgin has no substrate or superstrate, they can be referred to as two adstrates. \par
    Superstrates have more influence than adstrates and more than substrates. Adstrates refer to languages that are in a middle ground where the degree of influence is not that clear cut. \par
    A pidgin develops from a \textbf{pre-pidgin} to a \textbf{stable pidgin} to an \textbf{expanded pidgin}.

    \section{Creoles}
    \edfn{Creole}{A creole is a pidgin that has been passed down to the next generation, where the pidgin becomes the language of the people.}
    Typically, pidgins are short lived. However, some pidgins more on to transform into dynamic languages called \textbf{creoles}. This typically occurs when children are brought up speaking a pidgin as a first language.

    \exmp{Suriname}{Suriname began as a pidgin. The children of the people did not have their own language. When they were taught this pidgin, the status of the pidgin transitions into a creole---passed down to the next generation. The children were able to give the creole structure and syntax.\footnote{Related to critical period hypothesis---our linguistic functions are hard-wired to our brain, the skills of using and mastering language is innate. Some psycholinguists believe that there are parts of the brain which are dedicated to acquiring language (Chomsky's language acquisition device). Patients who have experienced brain damage at certain areas are unable to either produce language or comprehend language although they are able to do the other, or both.}}

        \subsection{Attitudes}
        Creoles are often regarded as inferior by speakers who use a standard form of the language. They are usually linked to slavery and subjection---primarily because most pidgins and creoles were products of colonisation.

        \subsection{Features of English-based Creoles}
        \begin{itemize}
            \item Expansion of vocabulary
            \item Grammatical structures that allow for the communication of more complex meanings
            \item Wider range of functions
            \item Reduplication of words (eg. \textit{small} \textrightarrow{} \textit{smalsmal})
        \end{itemize}

        \subsection{Formation of Creoles}
        The stages that a pidgin goes through to become a creole:
        \begin{enumerate}
            \item The formation of a pidgin through limited contact.
            \item The extension of a pidgin for nativisation.
            \item If the next generation of children are taught the pidgin as their mother tongue, there is further development which results in a creole.
            \item Creoles develop when there is a deliberate effort for it to be taught and used within a particular social context (linked to social status).
            \item The language can then have the potential to move towards a standard language with development in all linguistic aspects.
        \end{enumerate}
        
        \subsection{Creole Continuum}
        With respect to the superstrate,
        \begin{itemize}
            \item Acrolect: the most standard, least creole-like variety.
            \item Mesolect: the intermediate variety.
            \item Basilect: the most creole-like, the `purest' variety.
        \end{itemize}

    \section{Effects of Pidgins and Creoles}
    The indigenous languages may be forgotten or even become extinct if the pidgin/creole is successful, leading to loss of culture and meaning. The `invasion' of foreign words will also change linguistic choices of the people, such as change in pronunciation.

    While most pidgins originated from the need for a `middle language', they have evolved into markers of identity for some of their speakers, even if it is to establish their status as `outsiders'.
\end{document}
