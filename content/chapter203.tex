\documentclass[../main.tex]{subfiles}

\begin{document}
    \section{Language Change: Pidgins and Creoles}

    \subsection{Pidgins}
    \edfn{Pidgin}{A pidgin is created when two speakers with no common language have to communicate with each other.}
    Pidgins have been described as `incomplete' languages. \par
    They are formed when speakers of two different languages come into contact and have a need to communicate, but for limited purposes such as trade. \par
    Colonisation is an example of where pidgins can arise. \par
    
    The three Gs of colonisation:
    \begin{itemize}
        \item God
        \item Greed
        \item Glory
    \end{itemize}

    Three types of colonisation: 
    \begin{itemize}
        \item \textbf{Displacement:} substantial settlement (eg. North America)
        \item \textbf{Subjection:} sparse settlement (eg. Nigeria)
        \item \textbf{Replacement:} new labour from elsewhere (eg. Jamaica)
    \end{itemize}
    Most pidgins are based on European languages and this is a reflection of their history of colonisation.

    \exmp{Case Study: West Africa}{
    In the 19th century, the whole of Africa was shared out among European powers. Large black population of Africa was administered by a small group of British Officials. \par 
    Africa was supplying raw materials for the UK. The officials needed to communicate with the native population. As a result, Guinea Coast Creole English was the lingua franca in that region.
    }

    Some common features of English-based pidgins:
    \begin{itemize}
        \item distinguishable from dialects as they are clearly made up from two different language sources.
        \item simplified grammatical structures, primarily because it was for limited use.
        \item vocabulary dominated by English words.
        \item small vocabulary (700--2000 words).
        \item small range of linguistic functions.
    \end{itemize}

    \subsubsection{Structures of Pidgins}
    \edfn{Strate}{A strate is a language that inflences, or is influenced by another language due to contact.}
    A pidgin contains a substrate and a superstrate. If the pidgin has no substrate or superstrate, they can be referred to as two adstrates. \par
    Superstrates have more influence than adstrates and more than substrates. Adstrates refer to languages that are in a middle ground where the degree of influence is not that clear cut. \par
    A pidgin develops from a \textbf{pre-pidgin} to a \textbf{stable pidgin} to an \textbf{expanded pidgin}.

    \subsection{Creoles}
    \edfn{Creole}{A creole is a pidgin that has been passed down to the next generation, where the pidgin becomes the language of the people.}
    Typically, pidgins are short lived. However, some pidgins more on to transform into dynamic languages called \textbf{creoles}. This typically occurs when children are brought up speaking a pidgin as a first language.
\end{document}