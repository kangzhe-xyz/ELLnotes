\documentclass[../main.tex]{subfiles}

\begin{document}
	\section{History of English}
	\edfn{Sapir-Whorf Hypothesis}{Language is reflective of culture and thinking, and vice-versa.}
	Languages are a marker of culture, identity, and beliefs.
	Collocations are important for cultures because different languages will have different perceptions of certain concepts:
	\exmp{Norwegian}{"Faen" is a very vulgar word in the Norwegian language, which is referred to as "the Devil".}
	\exmp{Portugese}{Calling someone in Portugesem about a goat is very offensive.}

	\subsection{Middle English}
	Some elements of the norman culture has influenced Old English, which formed Middle English (ME). For example, some spelling conventions of the french language, such as adding diacritics to account for phonemes that cannot be represented by the Roman alphabet, had a huge influence on the development of the English language.

	\subsection{The Lexicon of English}
	Anglo-Saxon base: contained a of simple words such as  \par
	Scandanavian borrowings: Vikings began attaking the northern and eastern shores of Britain,  \par
	French borrowings: Norman conquests \par
	1066 - 1250: not muc \par
	English language is not immoral -- the contecpt were expanded when the french arrived.  \par
	Hosehold relationships - aunt, uncle, nephew, etc. plurality an always be \par
	1400 onwards - fewer core items and more items with an edge of sophistication: dance, April, native, etc. \par \par
	Latin borrowings: influenced the language of Germanic tribes even before they set foot in Britain.  \par
	Lation loanwords reflected the superior material culture of the Roman empire, which had spread across europe: street, wall, candle, chalk, inch. \par
	Latin was also the language of Christianity, and St Augustine arrived in Britain in AD 597 to Christianise the nation. Terms in religion were borrowed: pope, bishop, monk, cleric, demon, disciple \par
	Christianity also brought with it learning: circul, not (note), paper, scol (school) \par
	Latin was also the language of science and philosophy, and science came into the picture during hte early Modern English period. As scientific and philosophical empricism was beginning to be valued, many of the new words are academic in nature: \textit{apparatus, formula, equinox, subtract, vacuum, etc.} \par
	This resulted in the distiction between learned and popular vocabulary in English. \par
\end{document}