\documentclass[../main.tex]{subfiles}
 
\begin{document}
	\section{Semantics}
		\begin{preamb}
			Semantics is the study of meaning in a language. It is concerned with the relationships between words and words, words and language users, and words and referents.
		\end{preamb}
	
	\subsection{Denotation and Connotation}
	\edfn{Denotation}{Denotation is the lexical/primary definition of a word.}
	\edfn{Connotation}{Connotation is the emotive and imaginative association surrounding that word.}
	Denotations of a word are easy to determine; connotations are usually implied or are emotional in nature, they are harder to decipher. This might cause misunderstandings.
	
	\subsection{Semantic Relations Between Words}
	\subsubsection{Synonyms and Antonyms}
	\edfn{Synonym}{Two or more words are synonyms of each other if they have similar meanings.}
	\edfn{Antonym}{Two words are antonyms of each other if they have opposite meanings.}
	
	\subsubsection{Hyponomy and Meronymy}
	\edfn{Hyponym}{A word (A) is a hyponym of another word (B) if (A) is an element of (B).}
	\edfn{Meronym}{A word (A) is a meronym of another word (B) if (A) is a constituent of (B).}
	
	\subsection{Collocations}
	\edfn{Collocation}{Collocations are partly or fully fixed expressions that have become established through repeated context-dependent use.}
	\exmp{Collocation}{In the semantic domain of sport, one might see words like ``goal'', ``bracket'', ``sportsmanship'' \textit{etc.}}
	
	\subsection{Figurative Language}
	\begin{itemize}
		\item Idioms
		\item Alliteration (technically not semantics)
		\item Onomatopoeia
		\item Personification
		\item Simile
		\item Metaphors
		\item Hyperbole
		\item Clich\'{e}
	\end{itemize}

	\subsection{Semantic Change}
	\subsubsection{Change in Denotation}
	\edfn{Broadening}{Broadening is when a word becomes applicable in more contexts than it previously was and means more than it previously did.}
	\edfn{Narrowing}{Narrowing is when a word has a reduction in contexts it can appear in.}
	\edfn{Shift}{A work undergoes semantic shift if its meaning has completely changed.}
	\exmp{Change in Denotation}{\textbf{Broadening:} ``business'' used to only mean the state of being busy, now it means all kinds of work occupations \\ \textbf{Narrowing:} ``fowl'' used to refer to birds in general, now it only refers to farmyard hen.}
	
	\subsubsection{Change in Connotation}
	\edfn{Amelioration}{Amelioration is when the word has gained positive connotations/lost negative connotations.}
	\edfn{Pejoration}{Pejoration is when the word has lost positive connotations/gained negative connotations.}
	\exmp{Change in Connotation}{\textbf{Amelioration:} ``sick'' used to only refer to illness and disgust, now it can also mean something is outstandingly good. \\ \textbf{Pejoration:} ``accident'' used to mean a chance event, now it carries the connotation of misfortune.}
\end{document}