\documentclass[../main.tex]{subfiles}

\begin{document}
    \section{Speech Act Theory}

    \begin{preamb}
        When we speak, we are also simultaneously ``doing things'', as well as eliciting certain effects i.e. any utterance aims to do something.
    \end{preamb}

    \subsection{Locutionary Forces}
    Speech acts contains three components: 
    \edfn{Locutionary Act}{An act of uttering a sentence with a certain sense and reference, also known as ``meaning''.}
    \edfn{Illocutionary Act}{An act of performing some action in saying something (eg. informing, claiming, guessing, threathening, etc.)}
    \edfn{Perlocutionary Act}{What speakers bring about or achieve by saying something (eg. persuading, deterring)}
    Different people perceive utterances differently, so an utterance may have different perlocutionary forces.

    \exmp{Locutions}{``The final exam will be difficult...''---locutionary act \\ 
    ``...let me remind you''---illocutionary act \\
    The perlocutionary act here is that I may have convinced you to study harder for the final exam.}

    Turn taking: who gets to speak, and how long. Length of turn is the phenomenon study at the workplace conversations.
\end{document}