\documentclass[../main.tex]{subfiles}

\begin{document}
    \section{Speech Act Theory}

    \begin{preamb}
        When we speak, we are also simultaneously ``doing things'', as well as eliciting certain effects i.e. any utterance aims to do something.
    \end{preamb}

    \subsection{Locutionary Forces}
    Speech acts contains three components: 
    \edfn{Locutionary Act}{An act of uttering a sentence with a certain sense and reference, also known as ``meaning''.}
    \edfn{Illocutionary Act}{An act of performing some action in saying something (eg. informing, claiming, guessing, threathening, etc.)}
    \edfn{Perlocutionary Act}{What speakers bring about or achieve by saying something (eg. persuading, deterring)}
    Different people perceive utterances differently, so an utterance may have different perlocutionary forces.

	\exmp{Locutions}{\begin{itemize}
		\item Locutionary act: ``The final exam will be difficult...''
		\item Illocutionary act: ``...let me remind you''
		\item Perlocutionary act: I may have convinced you to study harder for the final exam.
	\end{itemize}}

    Turn taking: who gets to speak, and how long. Length of turn is the phenomenon study at the workplace conversations.

	\subsection{Gricean Maxims}
	The four maxims of (spoken) communication are: \begin{itemize}
			\item maxim of quality: truth value,
			\item maxim of quantity: appropriate (acceptably low) number of words,
			\item maxim of relevance: context cohesion, and
			\item maxim of manner: how it is said.
	\end{itemize}

		\subsubsection{Violation of Maxims}
		When a maxim is violated on purpose, it can cause communication breakdown. A repair sequence is sometimes inserted to bring the conversation back.

		\subsubsection{Flouting of Maxims}
		When a maxim is violated on purpose with a communication intent, it is said to be \textbf{flouted}. Flouting of maxims can form \textbf{implicatures}, where the literal meaning is not enough to keep the conversation going, or at least the way it was intended to.

\end{document}
