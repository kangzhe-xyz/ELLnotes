\documentclass[../main.tex]{subfiles}

\begin{document}
    \section{Language and Gender}
        \subsection{Prosodic Differences}
            \begin{itemize}
                \item Verbosity: Men were found to talk more in the following circumstances: \begin{itemize}
                    \item during faculty meetings (Eakins and Eakins, 1979)
                    \item academic electronic discussion groups (Herring et al, 1992)
                    \item classroom interactions (Swann, 1989)
                \end{itemize}
                \item Turn-taking and Interruptions: Men are more likely to interrupt women in mixed-gender interactions regardless of familiarity. (Eimmerman, West, 1975, 1977) (Eakins and Eakins, 1979) (James and Clarke, 1993).
                \item Back-channel Support and Minimal Responses: Women are more likely to give more back-channel support than men (Fishman, 1980) (DeFrancisco, 1981) (Preisler, 1986), with the most common being minimal responses.
                \edfn{Minimal Response}{A minimal response is a fragment of speech which can be a sound to signify something. Examples are sounds like ``m'', ``uh-huh'' etc. }
                Malts and Borker (1982) and Tannen (1990) suggest that minimal responses represent different things for the two genders---for men it was agreement; for women it was listenership.
                \item Hedging: THe common belief is that women tend to hedge (well, you know, like, I think) and use epistemic modal forms (should, could, would, might). All these forms are commonly known to function as mitigation. \par
                Robin Lakoff (1975) was the greatest proponent that women's language was tentative. Fishman (1980), Preisler (1986), Pichler (2009) confirm that women use more hedges than men. This does not mean that men do not hedge. A study by Holmes revealed that the expressions ``sort of'' and ``kind of'' were used more by men than women.
            \end{itemize}

        \subsection{Sexism in English}
            \subsubsection{Insult Terms}
            Some categories of words in the English language can be insulting: \begin{enumerate}
                \item body parts
                \item sexual behaviour (eg slut-shaming)
                \item appearance
                \item animals
            \end{enumerate}
            The connotations behind the feminine version of the words may hint to us on how sexist the English language is. 
            
\end{document}