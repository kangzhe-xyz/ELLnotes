\documentclass[../main.tex]{subfiles}

\begin{document}
    \section{Language and Gender}
        \subsection{Prosodic Differences}
            \begin{itemize}
                \item Verbosity: Men were found to talk more in the following circumstances: \begin{itemize}
                    \item during faculty meetings (Eakins and Eakins, 1979)
                    \item academic electronic discussion groups (Herring et al, 1992)
                    \item classroom interactions (Swann, 1989)
                \end{itemize}
                \item Turn-taking and Interruptions: Men are more likely to interrupt women in mixed-gender interactions regardless of familiarity. (Eimmerman, West, 1975, 1977) (Eakins and Eakins, 1979) (James and Clarke, 1993).
                \item Back-channel Support and Minimal Responses: Women are more likely to give more back-channel support than men (Fishman, 1980) (DeFrancisco, 1981) (Preisler, 1986), with the most common being minimal responses.
                \edfn{Minimal Response}{A minimal response is a fragment of speech which can be a sound to signify something. Examples are sounds like ``m'', ``uh-huh'' etc. }
                Malts and Borker (1982) and Tannen (1990) suggest that minimal responses represent different things for the two genders---for men it was agreement; for women it was listenership.
                \item Hedging: THe common belief is that women tend to hedge (well, you know, like, I think) and use epistemic modal forms (should, could, would, might). All these forms are commonly known to function as mitigation. \par
                Robin Lakoff (1975) was the greatest proponent that women's language was tentative. Fishman (1980), Preisler (1986), Pichler (2009) confirm that women use more hedges than men. This does not mean that men do not hedge. A study by Holmes revealed that the expressions ``sort of'' and ``kind of'' were used more by men than women.
            \end{itemize}

        \subsection{Sexism in English}
            \subsubsection{Insult Terms}
            Some categories of words in the English language can be insulting: \begin{enumerate}
                \item body parts
                \item sexual behaviour (eg slut-shaming)
                \item appearance
                \item animals
            \end{enumerate}
            The connotations behind the feminine version of the words may hint to us on how sexist the English language is. Feminine traits, for whatever reason, is attributed to be more insulting.

            \subsubsection{Asymmetry/Symmetry}
            This refers to what extent a lexical item can account for eg. person/human being vs man and woman. It is also evident that English is very much androcentric where hte male form is the norm and the female form is marked. 

            \subsubsection{Titles}
            Mr vs Miss/Mrs/Mdm/Ms: these titles are revealing the marital status of a woman, while for a man it does not matter. The existence of these titles also suggest that society viewed the marital status of women more important than men, and that there is an assumption that a Miss will eventually progress to a Mrs/Mdm. While there were attempts to change these titles such as Ms, Frau, Signora, the connotations are still extremely hard to remove in the psyche of the masses. 

            \subsubsection{Marked and Unmarked Terms}
            From a morphological standpoint, the suffixes ``-enne'', ``-ette'', ``-ess'' are understood to be feminine. \par
            Semantic derogation: bachelor vs spinster vs bacherlorette.

        \subsection{Explanations}
            \subsubsection{Deficit}
            Deficit is an idea proposed by Lakoff, where it is suggested that women's language contains defect due to their lack of access to power in history. Furthermore, societal discourse also shapes how women are perceived, and hence further amplifying this Deficit. 

            \subsubsection{Dominance}
            Language can reflect and perpetuate social and gender inequality. Men asserting dominance over women in interactions intensifies the power distance; studies by Itakura \& Tsui (2004) showed that women needed to fight harder than men to gain conversation control during interruptions.

            \subsubsection{Differences}
            It is possible that such traits are manifested in language due to how boys and girls are being socialised at young ages and are trained to communicate in different ways (eg. saying ``that's not very lady-like to say that'' to a young girl). Hence individuals are pressured by society to use language in accordance to the gender that is prescribed upon them by society. 

            \subsubsection{Social Constructionist}
            Proposed by Holmes (2007) and others, it states that individuals are aware of what constitutes masculine and feminine talk, and they express such identities according to their needs. Various labels to this approach are: post-modern (Cameron, 2005), performative (Butler, 1990), and dynamic (Caates, 2004). The theory looks at multiplicity and heterogeneity of masculinities and femininities. In essence, people speak differently in different situations in order to achieve different outcomes.
            
            \subsubsection{Communities of Practice}
            Proposed by Eckert and McConnell-Ginet (1992, 1995), communities allow for individuals to recognise interactional and social practices. In these communities, those who abide by the norms are rewarded, and those who deviate from the norms may face penalties. 

        \subsection{Some Observations in the Handouts}
        \begin{itemize}
            \item Aggressive prescription of the gender identity: ``[Y]ou're a woman.''
            \item Marked choices when describing certain articles: ``the way a woman shaves.''
            \item Typography: women's products use handwritten cursive scripts suggesting elegance, and men's products use bold sans-serif scripts suggesting strength.
            \item ``It's our closest shave for your closest moments.'' suggesting that women need the dependence on men.
            \item ``It's so \textit{cute} when a woman voices her opinion like it matters.'' patronising and derogatory.
        \end{itemize}

        \subsection{Is All Hope Lost?}
        no because you can use language for the other effect---strong is beautiful
\end{document}