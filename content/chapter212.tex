\documentclass[../main.tex]{subfiles}
\begin{document}
	\section{Language and Politics}
		\subsection{Unpacking Politics}
			Politics is related to politicians, governments, law-making, international relations, etc. \par
			Politics looks at social relationships which deal with power, governance, and authority. These can be formal, ``official structures'' (eg. a government) or those which are a bit more informal (eg. in a school).

			\subsubsection{What do Politicians Want?}
				\begin{itemize}
					\item Power, credibility, change (?), support
					\item In most democratic contexts, a \textbf{mandate}.
					\item The support of the electorate as a means of legitimising their power/authority.
					\item Therefore, one of the marks of a good politician is their ability to ..
				\end{itemize}

				What sort of strategies do politicians employ to persuade their audience?
					\begin{itemize}
						\item Choice of pronouns (1Pl ``our'').
						\item Demonising enemies, creating a common enemy (`us' vs `them' mentality).
						\item Code-switching to appeal to all groups (or as many as possible) (in a multilingual context).
					\end{itemize}
		
		\subsection{Techniques}
			\subsubsection{Catch Phrases}
				\begin{itemize}
					\item ``Yes we can'' Obama Campaign
					\item ``Our Lives, Our Jobs, Our Future'' PAP Campaign
				\end{itemize}
				
			\subsubsection{Metaphors and Similes}
			\exmp{Obama Speech}{\textbf{Metaphors:} \begin{itemize}
					\item My fellow citizens: I stand here today humbled by the task before us, grateful for the trust you've bestowed, mindful of the sacrifices borne by our ancestors.
					\item I thank President Bush for his service to our nation (applause) as well as the generosity and cooperation he has shown throughout this transition.
					\item Forty-four Americans have now taken the presidential oath. The words have been spoken during \textbf{rising tides of prosperity} and the \textbf{still waters of peace}. Yet, every so often, the oath is taken amidst \textbf{gathering clouds and raging storms}. At these moments, America has carried on not simply because of the skill or vision of those in high office, but because we, the people, have remained faithful to the ideals of our forebears and true to our founding documents.
			\end{itemize}}


			\subsubsection{Rule of Three}
				\exmp{GE2015}{``With you, for you, for Singapore'' slogan of the PAP during campaign.}

			\subsubsection{Parallelism}
				\exmp{Obama Speech}{\begin{itemize}
						\item ``On this day, we gather because we have chosen hope over fear, unity of purpose over conflict and discord.''
						\item ``On this day, we have come to proclaim an end to the petty grievances and false promises, the recriminations and worn out dogmas, that for far too long have strangled our politics.''
					\end{itemize}}

			\subsubsection{Euphemism}
				``Alternative facts'': euphemism for lying. \par
				Euphemisms are used to manage connotations, if something bad happens, use euphemism to describe it so it will influence the way the audience interprets the messages.

			\subsubsection{Dyseuphemism}
				``On September the 11th, the \textbf{enemies of freedom} committed an \textbf{act of war} against our country.'' --George Bush, 2001 \par
				In this one they are creating a common enemy, that threaten the fundamentals or the Constitution of the US.

			\subsubsection{Other Common Strategies}
				\begin{itemize}
					\item Hyperbole --- exaggeration 
					\item Repetition
					\item Rhetorical questions --- to get people to think about certain issues
				\end{itemize}

			%consider the concept of political correctness, and how that is considered and unpacked today. see: Verbal Hygiene
\end{document}
