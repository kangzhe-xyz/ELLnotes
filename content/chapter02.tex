\documentclass[../main.tex]{subfiles}
 
\begin{document}
	\section{Syntax I: Word Classes}
	\begin{preamb}
		Sentences sound grammatical because of syntax. This chapter will cover word classes, their attributes, and what constitutes to a grammatically sound sentence
	\end{preamb}
	\subsection*{Lexical Classes}
	\subsection{Nouns}
	\edfn{Nouns}{A noun is a word that refers to objects or ideas.}
	
	\subsubsection{Classification of Nouns}
	Nouns can be classified into four types: \textbf{proper} and \textbf{common} nouns, and \textbf{concrete} and \textbf{abstract} nouns.
	
	\begin{enumerate}
		\item \textbf{Proper nouns} are nouns that denote things that have a unique reference. 
		\item \textbf{Common nouns} refer to things in the general sense.
		\item \textbf{Concrete nous} refer to things that are tangible, or can be interacted with the five senses (visual, auditory, gustatory, olfactory, tactile).
		\item \textbf{Abstract nouns} refer to things that are intangible.
	\end{enumerate}

	\subsection{Verbs}
	\edfn{Verbs}{A verb is a word that denotes a process.}
	The way to check if a word is a verb is to see if it can be \textbf{inflected for tense}. 
	
	Every standard grammatical sentence contains at least \textbf{one verb}.
	
	\subsubsection{Tense and Aspect}
	\edfn{Tense}{Tense indicated the location of the event in time, with respect to the utterance.}
	Tense can be \textbf{past}, \textbf{present}, or \textbf{future}.
	\edfn{Aspect}{Aspect refers to the state of completion of the event.}
	Aspect can be \textbf{progressive} or \textbf{perfective}.
	\exmp{Tense and Aspect}{The ``past perfect'' refers to an event that occurred in the \textbf{past} (tense), and has been \textbf{completed} (aspect) before another event.}
	
 	\subsubsection{Classification of Verbs}
	Verbs can be classified in different ways. 
	
	\subsubsection*{By Syntax}
	\begin{itemize}
		\item Verbs that can only take one agent are \textbf{intransitive}.
		\begin{exe}
			\ex Julia talks.
			\ex[*] {Julia talks me.}
			\ex I sneezed loudly.
			\ex[*] {I sneezed loudly tissue paper.} 
		\end{exe}
		This means \textit{``talk''} and \textit{``sneeze''} are \textbf{intransitive}.
		\item Verbs that can take an agent and a recipient are \textbf{transitive}.
		\begin{exe}
			\ex He makes chairs.
			\ex[*] {He makes.}
			\ex Amy painted a lot when she was younger.
			\ex[*] {Amy painted.}
		\end{exe}
		This means \textit{``make''} and \textit{``paint''} are \textbf{transitive}.
	\end{itemize}
	
	\subsubsection*{By Function}
	\begin{itemize}
		\item \textbf{Main verbs} are verbs that \textit{state the main process} occurring in a verb phrase.
		\item \textbf{Auxiliary verbs} are verbs that serve a \textit{grammatical} function, denoting the \textit{tense}, \textit{aspect}, or \textit{modality}.
	\end{itemize}
	Auxiliary verbs usually exist with the main verb, except for the verbs ``have'', ``do'', and ``be'' which can exist as either auxiliary verbs or main verbs by themselves 

	\subsubsection*{By Processes}
	\begin{itemize}
		\item \textbf{Material processes} are when a participant takes a physical action.
		\item \textbf{Mental processes} express thoughts or feelings.
		\item \textbf{Verbal processes} are meant for communication.
		\item \textbf{Existential processes} indicate existence of the participant.
		\item \textbf{Relational processes} connect two different ideas.
		\item \textbf{Behavioural processes} involve the participants' physiological and psychological processes.
	\end{itemize}

	\subsubsection{Participants}
	\edfn{Participant}{A participant is an entity that is executing or is influenced by the process.}
	Participants can be human or non-human, concrete or abstract, specific or non-specific.
	
	\edfn{Agent}{An agent is one who executes the process.}
	\edfn{Recipient}{A recipient is directly influenced by the process the agent is executing.}
	
	\textbf{Agentisation} is the process of making an entity the agent. The intent is to give the meaning that the agent is the one taking action to carry out the process. 
	
	\subsubsection{Modality}
	\edfn{Modality}{Modality indicates the possibility, certainty, or obligation.}
	\exmp{Modality}{\begin{exe}
			\ex You may come to the party.
			\ex You must come to the party.
	\end{exe} 
	The modal ``must'' in (2) shows a higher obligation than the modal ``may'' in (1).}
	
	In modality analysis, it is important to consider the \textbf{constructed relationship} between the text producer and the text recipient.
	
	The process of intentionally lowering the modality of a statement is known as \textbf{hedging}. Hedging lowers the forcefulness of the sentence.
	
	\exmp{Hedging}{\begin{exe}
			\ex The sun sets in the west.
			\ex The sun sets in the west, I think.
	\end{exe}
	In example (2), ``I think'' is hedging as it lowers the confidence of the speaker about whether the sun sets in the west or not.}

	\subsubsection{Analysis of Verbs}
	Three things are to be considered when analysing verbal features of a text:
	\begin{enumerate}
		\item \textbf{Participants:} what are the entities involved?
		\item \textbf{Roles:} what roles do the participants play with regards to the processes?
		\item \textbf{Attitude:} what does the usage of verbs tell about the attitude of the text producer?
	\end{enumerate}

	\subsection{Adjectives and Adverbs}
	\edfn{Modifier}{A modifier is a word that narrows down the scope of a certain reference.}
	\edfn{Adjective}{Adjectives talk about qualities or characteristics of referents.}
	Adjectives usually appear before nouns.
	
	Adjectives exist in three degrees:
	\begin{enumerate}
		\item \textbf{Adjective:} the root word itself.
		\item \textbf{Comparative:} used when comparing the qualities of two referents.
		\item \textbf{Superlative:} used to identify the extreme in a collective group of referents.
	\end{enumerate}
	
	\edfn{Adverbs}{Adverbs modify verbs, adjectives, or other adverbs.}
	Adverbs can answer: how, when, where, why, or to what extent---how often or how much.
	
	Adverbs are usually taught to describe verbs but they can describe other things too.
	
	There are three kinds of adverbs:
	\begin{enumerate}
		\item \textbf{Circumstance adverbs:} they talk about manner, time, frequency, and place.
		\item \textbf{Degree adverbs:} they talk about the intensity of the described.
		\item \textbf{Sentence adverbs:} they express the text producer's attitude to the content in a sentence.
	\end{enumerate}
	
	\subsection*{Grammatical Classes}
	\subsection{Determiners}
	\edfn{Determiners}{Determiners, like adjectives, modify the noun and narrows down the range of referents \textbf{but does not attribute qualities}.}
	Determiners always precede nouns.
	\subsubsection{Types of Determiners}
	\begin{enumerate}
		\item \textbf{Articles:} \begin{enumerate}
			\item \textbf{Indefinite:} makes a generic reference. (a, an)
			\item \textbf{Definite:} indicates an unambiguous referent. (the)
		\end{enumerate}
		\item \textbf{Possessive:} denotes possession. (his, her, their, our, my, its)
		\item \textbf{Demonstrative:} ``points'' at things. (this, that, these ,those)
		\item \textbf{Quantifiers/Numbers:} quantifies referents. (one, two, two hundred and twenty-six)
	\end{enumerate}

	\subsection{Pronouns}
	\edfn{Pronouns}{Pronouns are used to replace nouns.}
	Pronouns appear at places nouns would be.
	
	The process of converting a noun to a pronoun is known as \textbf{pronominalisation}.
	
	\subsubsection{Types of Pronouns}
	\begin{enumerate}
		\item \textbf{Possessive:} denotes possession but it replaces a \textit{noun phrase}. \begin{exe}
			\ex \begin{xlist}
				\ex This is \textit{my book}.
				\ex This is \textit{mine}.
			\end{xlist}
		\end{exe}
		\item \textbf{Demonstrative:} 
		\begin{exe}
			\ex \begin{xlist}
				\ex \textit{This book} is mine.
				\ex \textit{This} is mine.
			\end{xlist}
		\end{exe}
	\end{enumerate}
	
	\begin{center}
		\begin{tabular}{rccc}
			\hline \hline
			& \textbf{1st person} & \textbf{2nd person} & \textbf{3rd person} \\ 
			\hline 
			\textit{sg.} & I & You & He/She/It \\ 
			\textit{pl.} & We & You & They \\ 
			\hline \hline
		\end{tabular} 
	\end{center}

	\edfn{Synthetic Personalisation}{Synthetic personalisation is when the text recipient feels personally addressed.}
	
	\subsubsection{``We''}
	``We'' can be either \textit{inclusive} or \textbf{exclusive}.
	\begin{exe}
		\ex We've just won the lottery!
	\end{exe}
	The above example has an ambiguous ``we'', wherein ``we'' might include the text recipient and the text producer (inclusive), or only the text producer and their friends, but \textit{not the text recipient} (exclusive).
	
	\subsection{Conjunctions}
	\edfn{Conjunctions}{Conjunctions connect two clauses. They also express the relationship between clauses.}
	\edfn{Forefronting}{Forefronting is a strategy that moves fragments of sentences to the front. It is usually used for emphasis.}
	
	There are two types on conjunctions:
	\begin{enumerate}
		\item \textbf{Coordinating conjunctions:} \begin{itemize}
			\item joins clauses of equal importance;
			\item removing the conjunction will leave 2 independent clauses without a change in meaning.
		\end{itemize}
		\item \textbf{Subordinating conjunctions:} \begin{itemize}
			\item shows the relationship between the clauses
			\item removing the conjunction will alter the expressed relationship
		\end{itemize}
	\end{enumerate}
	
	\edfn{Subordinate Clauses}{Subordinate clauses are fronted by a subordinate conjunction. They cannot stand alone as a sentence because it does not provide a complete thought.}
	
	\subsection{Prepositions}
	\edfn{Prepositions}{Prepositions are words that denote place, manner, time, or circumstance, but do not attribute any further meaning to the main phrase.}.
	Prepositions exist in prepositional phrases (PP).
\end{document}
