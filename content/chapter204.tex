\documentclass[../main.tex]{subfiles}

\begin{document}
    \section{Dialectology}
    \subsection{Linguistic Studies}
        In dialectology, the main goals of the discipline are:
        \begin{itemize}
            \item to understand how language can vary according to regional and sociological factors,
            \item how can these changes be tracked and interpreted, and
            \item the significance in development of language.
        \end{itemize}

        \exmp{Linguistic Survey of India}{Conducted by Sir George Grierson from 1894--1928. \par
        
        \textbf{Methodology}\par
        Participants were asked to (i) read a standard passage in their language and (ii) translate a few phrases that were used in daily life. \par

        \textbf{Results}\par 
        Grierson found that there were 179 languages and 544 dialects spoken in India. \par

        \textbf{Criticisms} \par
        The participants only involved teachers and officials, and not all areas of india were represented. The workers hired to collect data may not be as sufficiently trained either.
        }

        
\end{document}