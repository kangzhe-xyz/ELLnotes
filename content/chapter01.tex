\documentclass[../main.tex]{subfiles}
 
\begin{document}
	\section{Morphology}
	\begin{preamb}
		Morphology is the study of shapes. This chapter will tell us what words are made up of and how words are formed.
	\end{preamb}

	\subsection{Morphemes}
	\edfn{Morpheme}{A morpheme is the smallest unit of meaning in a language. They contribute to the overall meaning of a word.}
	
	A word must contain \textbf{at least one morpheme}. Furthermore, the morpheme must be consistent across the language to be called one (productive), being used. 
	
	\enot{Morpheme}{A morpheme is denoted with curly brackets \{ \quad \} around it.}{\{morpheme\}}
	
	Morphemes exist in two forms: \textit{free} morphemes and \textit{bound} morphemes.
	
	\edfn{Free Morpheme}{A free morpheme is one that can stand on its own.}
	
	\exmp{Free Morpheme}{\begin{exe}
			\ex \{car\}
			\ex \{run\}
			\ex \{caterpillar\}
	\end{exe}}
	
	\edfn{Bound Morpheme}{A bound morpheme is one that cannot stand on its own. It needs to be attached to another morpheme to be able to form a word.}
	
	\exmp{Bound Morpheme}{\begin{exe}
			\ex \{un-\}
			\ex \{-ed\}
			\ex \{anti-\}
	\end{exe}}
	
	\enot{Combination of Morphemes}{Morphemes combined together to form a word are written in order with plus [+] signs between them.}{\{hand\} + \{bag\} + \{-s\} = handbags}
	
	\textbf{Note.} Spelling is arbitrary. The characters in the morpheme need not form the actual spelling of the word.
	
	\subsection{Morphological Processes}
	\subsubsection{Affixation}
	\edfn{Affixation}{Affixation is the process of attaching bound morphemes to stems.}
		
	\edfn{Stem}{A stem is a morpheme or a combination of morphemes that exist as a free form.}
	
	\exmp{Affixation}{\begin{exe}
			\ex \(\underbrace{\text{\{run\}}}_{\text{stem}} + \underbrace{\text{\{-ing\}}}_{\text{affix}} = \text{running}\)
			\ex \(\underbrace{\text{\{un-\}}}_{\text{affix}} + \underbrace{\text{\{clean\}}}_{\text{stem}} = \text{unclean}\)
			\ex \(\underbrace{\text{\{en-\}}}_{\text{affix}} + \underbrace{\text{\{light\}}}_{\text{stem}} + \underbrace{\text{\{-en\}}}_{\text{affix}}= \text{enlighten}\)
		\end{exe}
	(1) uses a suffix, (2) uses a prefix, (3) uses a circumfix.} 
	
	Affixation is sequential. Only one affix is added to a stem at one time.
	
	Affixes can be classified by their \textit{functions}. They are:
	\edfn{Derivational Affix}{A derivational affix changes the meaning of word class of the stem.}
	
	\edfn{Inflectional Affix}{An inflectional affix changes the tense, number, or aspect of the stem.}
	
	\subsubsection{Compounding}
	\edfn{Compounding}{Compounding is the formation of a new word by combining two or more stems.}
	
	\edfn{Headed Compounds}{Headed compounds consist of a head and a modifier; the head contributes the core meaning of the compound and the modifier narrows down the meaning of the head.}
	
	\exmp{Headed Compounds}{\begin{exe}
		\ex	\(\underbrace{\text{\{hand\}}}_{\text{modifier}} + \underbrace{\text{\{bag\}}}_{\text{head}} = \text{handbag}\)
		\ex \(\underbrace{\text{\{love\}}}_{\text{modifier}} + \underbrace{\text{\{sick\}}}_{\text{head}} = \text{lovesick}\)
		\end{exe}}
	
	\subsubsection{Conversion}
	\edfn{Conversion}{Conversion is the formation of a new word by changing its word class without modification to its form.}
	
	\edfn{Nominalisation}{Nominalisation is the process of converting a word into a noun.}
	Nominalisation has effects of:
	\begin{itemize}
		\item increased formality
		\item makes things sound abstract/sophisticated
		\item conceptualisation of things
	\end{itemize}
	
	\edfn{Verbalisation}{Verbalisation is the process of converting a word into a verb.}
	Verbalisation has effects of:
	\begin{itemize}
		\item reduces formality
		\item makes things sound more tangible
		\item sounds more active
	\end{itemize}
	\exmp{Conversion}{In the case for nominalisation, \[\text{run (v)} \rightarrow \text{run (n)}\] In the case for verbalisation, \[\text{tag (n)} \rightarrow \text{tag (v)}\]}
	
	\subsubsection{Other Morphological Processes}
	\subsubsection*{Clipping}
	\edfn{Clipping}{Clipping is the process of shortening a word.}
	\exmp{Clipping}{\begin{exe}
			\ex in\textbf{flu}enza \(\rightarrow\) \textit{flu}
			\ex re\textbf{frige}rator  \(\rightarrow\) \textit{fridge}
	\end{exe}}
	Attributes of clipped words:
	\begin{itemize}
		\item affixes are omitted
		\item stressed syllables are selected
	\end{itemize}
	
	\subsubsection*{Acronomy}
	\edfn{Acronomy}{Acronomy is the process of shortening long word(s) into short pronounceable strings.}
	\exmp{Acronomy}{\begin{exe}
			\ex radar
			\ex laser
			\ex sonar
	\end{exe}}
	
	\subsubsection*{Blending}
	\edfn{Blending}{Blending is the process where two words are fused together.}
	\exmp{Blending}{\begin{exe}
			\ex breakfast + lunch = brunch
			\ex reduction + oxidation = redox
			\ex smoke + fog = smog
	\end{exe}}
	Blending takes place at a \textit{similar} vowel (or closest).
	
	\subsubsection*{Borrowing}
	\edfn{Borrowing}{Borrowing is taking words from other languages and adding it to its own lexicon.}
	\exmp{Borrowing}{\textbf{Loanwords} are words borrowed directly from another language 
	\begin{exe}
		\ex kindergarten
		\ex calque
	\end{exe}
	\textbf{Calques} are words borrowed from another language by word-for-word or root-for-root translation. \begin{exe}
		\ex lose face
		\ex cookie
\end{exe}}
\end{document}