\documentclass[../main.tex]{subfiles}

\begin{document}
    \section{Paper 2 Essay}
    
    \subsection{General Guidelines}
    Do these when attempting an essay question\footnote{There is no general formula to writing a paper 2 essay, these are just the essential things to consider---there is no one size fits all approach.}:
        \begin{enumerate}
            \item Read the question carefully and identify the key word(s) that will reveal to you the \textbf{topic} to be discussed in the essay.
            \item Read the text and identify \textbf{examples} that are \textbf{significant} and \textbf{relate} to the theme of the question.
            \item \textbf{Analyse} the information in the text and \textbf{link} it to what you have learnt about language and society thus far.
        \end{enumerate}
        \exmp{Essay Planning}{Imagine you have a text that discusses the innovation of press printing which was introduced by Caxton, and the question asks you to talk about what this tells us about human nature and language. \par Here are some things worth discussing:
        \begin{itemize}
            \item The invention of printing press allowed for texts and manuscripts to be mass produced, which led to increase in literacy and more interest in reading and writing.
            \item The spirit of innovation that people hae drive them to improve on current situations and become more efficient in solving problems.
            \item Today, this spirit is still strong as people are still looking for more efficient ways to communicate, from using their smartphones to using platforms like Skype, video calling, etc.
        \end{itemize}
        }

    \subsection{Structure}
    Expectations:
    \begin{itemize}
        \item Continuous prose.
        \item PEEL structure where appropriate.
        \item Solid English.
        \item Bring in knowledge from Paper 1 where possible and appropriate.
        \item Bring in your own personal knowledge of language issues\footnote{It is better to present facts than personal anecdotes due to the nature of the discipline. Keep up-to-date with all domains as far as possible.}.
        \item Remember, it is a \textbf{sociolinguistics} paper.
    \end{itemize}
    
    \subsection{Body Paragraphs}
        Number of points: try to come up with three overarching ideas. Intro, three body paragraphs, conclusion. 
        \begin{itemize}
            \item Possible overarching themes for essay organisation.
            \item Elements from Paper 1.
            \item Potential linguists/case studies to quote.
            \item Examples beyond the texts.
        \end{itemize}

\end{document}