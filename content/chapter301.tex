\documentclass[../main.tex]{subfiles}

\begin{document}
    \section{Paper 2 Essay}
    
    \subsection{General Guidelines}
    Do these when attempting an essay question\footnote{There is no general formula to writing a paper 2 essay, these are just the essential things to consider---there is no one size fits all approach.}:
        \begin{enumerate}
            \item Read the question carefully and identify the key word(s) that will reveal to you the \textbf{topic} to be discussed in the essay.
            \item Read the text and identify \textbf{examples} that are \textbf{significant} and \textbf{relate} to the theme of the question.
            \item \textbf{Analyse} the information in the text and \textbf{link} it to what you have learnt about language and society thus far.
        \end{enumerate}
        \exmp{Essay Planning}{Imagine you have a text that discusses the innovation of press printing which was introduced by Caxton, and the question asks you to talk about what this tells us about human nature and language. \par Here are some things worth discussing:
        \begin{itemize}
            \item The invention of printing press allowed for texts and manuscripts to be mass produced, which led to increase in literacy and more interest in reading and writing.
            \item The spirit of innovation that people have drives them to improve on current situations and become more efficient in solving problems.
            \item Today, this spirit is still strong as people are still looking for more efficient ways to communicate, from using their smartphones to using platforms like Skype, video calling, etc.
        \end{itemize}
        }

    \subsection{Structure}
    Expectations:
    \begin{itemize}
        \item Continuous prose.
        \item PEEL structure where appropriate.
        \item Solid English.
        \item Bring in knowledge from Paper 1 where possible and appropriate.
        \item Bring in your own personal knowledge of language issues\footnote{It is better to present facts than personal anecdotes due to the nature of the discipline. Keep up-to-date with all domains as far as possible.}.
        \item Remember, it is a \textbf{sociolinguistics} paper.
    \end{itemize}
    
    \subsection{Body Paragraphs}
        Number of points: try to come up with three overarching ideas. Intro, three body paragraphs, conclusion. 
        \begin{itemize}
            \item Possible overarching themes for essay organisation.
            \item Elements from Paper 1.
            \item Potential linguists/case studies to quote.
            \item Examples beyond the texts.
        \end{itemize}

    \subsection{Feedback}
        \subsection*{T3W3 Timed Practice}
        [N17/II/2] Discuss in detail the range of attitudes to Singapore Standard English and Singapore Colloquial English, and explore some of the reasons for these attitudes. \hfill [25]

        \subsubsection*{General Observations}
        \begin{itemize}
            \item Work quickly but accuracy should not be compromised.
            \item Many could not finish the essays/last few bits of essay were rushed.
            \item Missing components (eg no intro, no conclusion).
            \item Describing and narrating (regurgitation of) examples are not enough, does not show engagement with the question.
            \item Language: exercise caution with grammar.
        \end{itemize}

        \subsubsection*{Task Analysis}
        \begin{itemize}
            \item Unpack the attitudes people may have to the language varieties; note that there are usually two extremes: receptive vs reactionary.
            \item \textbf{Text B(i)} \begin{itemize}
                \item Identify SGEM as a move to try to clamp down SCE among Singaporeans.
                \item Discuss in the early years of introduction of SGEM, quote Lee Kuan Yew calling SCE as a ``handicap''.
                \item The more alert ones were able to identify that ``relentless'' (line 5) was quite strong, and rejustified the claim about these government efforts are allowances.
                \item Allowances in relation to how certain SCE sayings or utterances or lexis specific to SCE making appearances in texts, broadcasts, or mainstream media.
                \item ``[M]ajor obstacle... global competitiveness'' (line 6--7).
                \item Gupta referring to SCE as a `step-tongue' (line 9); implying some attitudes.
                \item Heavy utilisation of SCE (line 11), discuss on how SCE is commonly used in daily situations.
                \item Structurally, SCE is a vernacular.
                \item Discuss efficiency of language: SCE like Internet language have the appeal of using far fewer words to convey the same idea, and fewer worries for grammatical considerations such as verb inflections. 
                \item English education (lines 13--14): alternative proposal of some linguists: recognise that SCE exists, and teach the structures of both SSE and SCE so that students can differentiate between the two varieties. 
                \item Prestige/marker of identity/solidarity and rapport among Singaporeans/for informal and comfortable interactions with one another (lines 15-20).
                \item Diglossia in Singapore: with clear distinction between one variety and another, there will be very defined roles for each variety.
                \item Code/style switching: not everyone can code-switch easily (also in Text B(ii) line 12--13).
            \end{itemize}
            \item \textbf{Text B(ii)} \begin{itemize}
                \item Talking about being able to sell overseas vs making programmes being relatable to Singaporeans.
                \item Perceiving language as a commodity vs using language for the establishment of solidarity.
                \item ``[L]ess sophisticated...'' (line 15) to describe SCE; representative of certain attitudes towards SCE
            \end{itemize}
        \end{itemize}
		
		\subsection*{T3 Block Timed Practice}
			\subsubsection*{Texts}
				\begin{itemize}
					\item \textbf{Text B(i)} \begin{itemize}
						\item Lines 21--26: note the choice of how CCS is written as the leader, where in the other direction of taking cues from Pereira could also be as valid. This is consistent with the androcentric nature of this text.
						\item ``Most fashionable females'' but there are quite a few pictures of men.
					\end{itemize}
				\item \textbf{Text B(ii)} \begin{itemize}
						\item Difference in the gender discussed (vs Text B(i))
						\item Line 1: ``Top Five'' discusses the aesthetic value they bring across. The criteria used to evaluate the men seems to be quantifiable. Allude to how the communication of males (competitive) is this discourse.
						\item Lines 3--4: ``has also set some hearts throbbing and temperatures rising.'' Some kind of sexual undertone here? Temperature rising is quite literally a physiological response to seeing something attractive. Some kind of ``primal attraction'' toward these people.
						\item Most of the pictures in this text are solo portraits, but in Text B(i) are group photos. Perhaps discuss the air time allocated to women.
						\item Lines 9--10: ``crisp and sharp'': crisp like a freshly ironed shirt, sharp as in smart. ``even at the age of 40'': expectation that with age one might not be as sharp (though it might not be as significant).
						\item ``Poised for greater things in the party'' (line 11) ``serving... over a decade'' (line 13): credentials mentioned that are beyond the intended scope (at least according to the headline) of the article.
						\item ``brain and brawn'' (line 18): given a holistic representation of Lim as an individual (though how he has ``brawn'' is not mentioned).
						\item ``handled himself eloquently'' (line 16) description of his behaviour. 
						\item No MP track record but many mentions of how they have natural flair and assume these positions of leadership (perceptions and attitudes). In contrast to the women in Text B(i), there was no mention of their track records.
						\item ``SIA pilot'' (line 20) credentials mentioned again, not a bad job in the context in Singapore. 
						\item ``[D]abbled with his own business ventures'' (line 22) he could have failed? The connotation of ``dabbled'' is not positive (though that might not be the intention here).
						\item ``[S]wooped into the hearts and minds'' (line 21) weird choice of word. Significant in the sense that it is alluding to the attractiveness of the candidate. Interesting how ``minds'' are mentioned, when the scope of an article is about ``hunks''.
						\item ``Topless photos'' (line 28) alluding to the concept of sexual attraction. Though, this is one of the thing that is related to the ``hunk'' theme of the article's headline.
						\item Quite incoherent with the lack of a headshot of Chee in the article.
						\item ``[R]un triathlons'' (lines 31--32) discusses Chee's healthy hobby.
					\end{itemize}
				\end{itemize}

		\subsection*{Preliminary Examinations}
			\subsubsection*{Section A}
				\begin{itemize}
					\item \textbf{Question 1} \begin{itemize}
						\item \textbf{Text A(i)} \begin{itemize}
							\item Prescriptivist attitudes: calling a language ``good'' and ``bad'', ``right'' and wrong''.
							\item Elaboration on the concept of possibly being incorrect/wrong.
							\item Shameful to use Singlish, as Singlish entails the lack of ability to develop fluency in StdE.
							\item Diglossic nature of SG, writer thinks that Singlish is the L-variety of the diglossia.
							\item Lines 9--10: cross-refer to Vivian Ler's study where particles have been successful in carrying meaning or nuance, acting as a kind of prosody, as opposed to what the writer has written here.
							\item Lines 11--12: ``class background'' and ``performativity'': an element at play when it comes to politics.
							\item Line 15: suggesting that Singlish is in fact manipulative.
							\item Line 16: ``comedic purposes''---Singlish is a joke.
							\item Lines 27--28: Code-switching is ``condescending'' and is under the shroud of ``false camaraderie''. Albeit contentious it is still a valid argument.
						\end{itemize}
						\item Text A(ii) \begin{itemize}
							\item Line 28: ``nation building''
							\item Line 40: ``insular and inhospitable to foreigners''
							\item SGEM mentioned
							\item Coxford dictionary
							\item Mostly standard points about identity and solidarity, and how Singlish is efficient and practical for Singaporeans.
						\end{itemize}
					\end{itemize}

					\item \textbf{Question 2} \begin{itemize}
						\item Text B(i) \begin{itemize}
							\item Line 10: ``deteriorating''---any other variety other than StdE is considered to be worse .
							\item Punctuation, syntax, emojis.
							\item Lines 29--36: There is no one size fits all approach to language production; it is something to be considered on a case by case basis.
						\end{itemize}
						\item Text B(ii) \begin{itemize}
							\item Line 14: `pictorial symbols' are not something new.
							\item Line 15: `similarity to the hieroglyphs of ancient Egypt.'
							\item Lines 18--30 suggest how the writer thinks about emoji.
							\item Line 23: `far more subjective'---the subjectivity that comes with integrating visual elements. However it is not good to jump into the conclusion that subjectivity does not exist in texts preceding emojis/visual elements integrated to written text due to emoji (Gricean maxims).
							\item Origins of emoji in Japan.
							\item Line 44: `Eggplant Emoji'---rather out of place given the high formality in the first half of the text.
							\item Line 47: `This makes emojis ideographs'---shows that the symbols represent an idea, and is more open to interpretation (which is her assertion throughout the piece).
							\item Second picture on second page: added emojis has a rather clear reference, and does not always replace text/word-for-word.
						\end{itemize}
						\item \textbf{Important things:} Communities of practice, multimodality, what people want when they use language, rapid evolution of language due to the fast pace of technology.
					\end{itemize}
					
					\item \textbf{Question 3} \begin{itemize}
						\item Text C(i) \begin{itemize}
							\item Adjectives: `scary' STDs, and `honest and informative' discussions.
							\item Layout: Left and right shows contrast between the two paradigms.
							\item Verbosity: large block of text on the right looks like an important issue, short words on the left suggests that the issue is being downplayed to a trivial matter.
						\end{itemize}
						\item Text C(ii) \begin{itemize}
							\item Line 1: `no sexy time'---intimate register, informal expression.
							\item Line 2: `emotionally-stunted young adults'---reflects the views on relationships and consent in Singapore.
							\item The way people are to be educated on consent
							\item Scenario presented taking into account how real-life situations can occur.
							\item The writer shoes that the efficacy of these sex education attempts through these modules is questionable at best (not graded, response of NUS spokesperson is laden with jargon).
						\end{itemize}
					\end{itemize}
			
					\item \textbf{Question 4} \begin{itemize}
						\item Text D(i) \begin{itemize}
							\item Line 1: `Fighters of the Nation'---heroic/idealistic figure pictured here.
							\item Line 8: Politicisation of ethnic identity.
							\item Line 8: `[C]leanse the country of corruption'---intention of the politician is if you do not subscribe to their political ideologies, you are corrupt.
							\item Abstract references such as dignity and rights.
						\end{itemize}
						\item Text D(ii) \begin{itemize}
							\item Line 10: `Is their country still Malay?'---the ideals of Malaysia. Also shows that Dr M thinks that the country (Singapore) used to be a Malay country.
							\item Possibly the demonisation of Dr M.
						\end{itemize}
						\item Text D(iii) (response to Text D(ii)) \begin{itemize}
							\item Importance of dissolving tensions, highlighting that Dr M's strategy of demonisation of Singapore.
							\item At the same time, he is showing that the tone of Text D(ii) does not uphold the ideals of Singapore (antagonisation of Dr M).
							\item Dysphemism---writing things a bit too critically than how things actually are.
							\item Usage of emojis and hashtags (though this might not be as relevant).
						\end{itemize}
					\end{itemize}
				\end{itemize}
\end{document}
