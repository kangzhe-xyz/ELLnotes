\documentclass[../main.tex]{subfiles}

\begin{document}
	\section{Syntax II: Constituency}
		\begin{preamb}
			Some assumptions in this section: \begin{enumerate}
				\item Words must occur in a particular order.
				\item Certain sequences of words \begin{enumerate}
					\item can be replaced by a single word;
					\item can be moved in a sentence;
					\item can ``hang together''.
				\end{enumerate}
				\item Particular slots (in sentences) must be filled by particular types of words or word sequences.
			\end{enumerate}
		\end{preamb}
		\subsection{Constituency}
		\edfn{Constituent}{A constituent is a group of units of the same linguistic type that form a longer unit of a different type.}
		For example, morphemes make words, words make phrases, phrases make clauses, clauses make sentences, etc.

		There exists a few constituency tests to determine whether a thing is actually a constituent.
		\begin{enumerate}
			\item \textbf{Substitution:} A group of words may be replaced by a single word.
			\item \textbf{Movement:} A group of words may appear in different positions in different versions of a sentence.
		\end{enumerate}
		\exmp{Constituency Test}{{\bf Substitution:} ``I am {\it Titanium}'' \(\to\) ``I am {\it Strontium}'' \\ Because this substitution is valid, ``{\it Titanium}'' is indeed a constituent. \\ {\bf Movement:} ``John wept {\it inconsolably}.'' \(\to\) ``{\it Inconsolably}, John wept.'' \\ Because the movement``{\it inconsolably}'' still results in a grammatical sentence, it is indeed a constituent.}

		\subsection{Phrases}
		\edfn{Phrases}{Phrases consist of a head and an optional modifier(s). Phrases are sequences of words or a group of words arranged in a grammatical construction, and functions as a unit.}

		There are five types of phrases:
		\begin{itemize}[leftmargin=.5in]
			\item[\bf NP:] Noun phrases; (Det) (Adj)* N
			\item[\bf VP:] Verb phrases; V (NP) (PP) (Adv)
			\item[\bf AdjP:] Adjective Phrases; (Adv) Adj (PP)
			\item[\bf AdvP:] Adverb Phrases; Adv
			\item[\bf PP:] Prepositional phrases; (P) (NP)
		\end{itemize}
		Constituency trees can be a good graphical method to visualise the different constituents in a sentence:
		\begin{center}
			\Tree [.S [.NP Writing ] [.VP [.V is ] [.NP fun ] ] ]
		\end{center}

		\subsection{Grammatical Function of Phrases}
		The acronym ``SVOCA'' can be used to describe phrases by their function rather than their form.
		\begin{itemize}
			\item[{\bf S:}]Subject: a participant
			\item[{\bf V:}]Verb: the process
			\item[{\bf O:}]Object: another participant
			\item[{\bf C:}]Complement: more information about subject/object
			\item[{\bf A:}]Adverbial: more information about the event
		\end{itemize}
		\exmp{SVOCA Analysis}{\undertext{My father}{S}, \undertext{the president}{Cs}, \undertext{shot}{V} \undertext{the sheriff}{O} \undertext{quietly}{Adv.}.}
\end{document}
