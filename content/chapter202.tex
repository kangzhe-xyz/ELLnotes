\documentclass[../main.tex]{subfiles}

\begin{document}

\section{Language Variants, Standard Languages, National Languages, and Venaculars}

    \subsection{Definitions}
    \edfn{Vernacular Language}{aaaaaaaa}
    \edfn{National Language}{}
    \edfn{Standard Language}{}
    \edfn{Lingua Franca}{}


    \subsection{Standard Languages}
    See above for the definition.

        \subsubsection{Selecting a Standard Language}
        The process of selecting a language as the standard language is as follows:
        \begin{enumerate}
            \item Determination: selecting the language
            \item Codification: acknowledging the existence of language
            \item Elaboration: formalising the language with regards to usage rules
            \item Implementation: making the people use the language
        \end{enumerate}

        \subsubsection{Judgement}
        A speaker would make judgements about certain constructions. There are a few ways people can judge things with regards to standard language:
        \begin{enumerate}
            \item \textbf{Correctness:} Correct or incorrect (spelling, grammar, punctuation)
            \item \textbf{Appropriateness:} Appropriate or inappropriate (context)
            \item \textbf{Usefulness:} Useful or not useful (function and practicality)
            \item \textbf{Beauty:} Beautiful or ugly (accents, styles) \footnote{Anyone who wishes to become a good writer should endeavour, before he allows himself to be tempted by the more showy qualities, to be direct, simple, brief, vigorous, and lucid. This general principle may be translated into practical rules as follows: prefer the familiar word to the far-fetched, the concrete word to the abstract, the single word to the circumlocution, the short to the long, the Saxon word to the Latinate. (Fowler, 1926)}
            \item \textbf{Social acceptability:} Socially acceptable or unacceptable
            \item \textbf{Offense:} Offensive or not (race, gender)
            \item \textbf{Controversy:} politicalisation of language
        \end{enumerate}

        \exmp{Who vs whom}{Grammar books are clear on `who' being a subject pronoun and `whom' being an object pronoun. 
        \begin{exe}
            \ex The boy who died.
            \ex The boy whom death approached.
        \end{exe}
        However, `whom' is generally underused by most people today, but the communication is still successful. So much so that the constructions that would use `whom' in standard language rules is equally acceptable if `who' is used.}

        \exmp{British (BrE) and American (AmE)}{It is known that many words in BrE and AmE are different though they refer to the same sign.
        \begin{exe}
            \ex flat (BrE), apartment (AmE)
            \ex rubbish (BrE), trash (AmE)
        \end{exe}
        Choices are made based on the individual's experience with the standard language they use.
        }

        \textbf{Social Judgements:} the rich and powerful vs poor and powerless, with regards to the hierarchy set by society.

        \textbf{Political Correctness:} terms that are racist, sexist, ageist, or those that show bias against creed and physical ability. 
        \begin{exe}
            \ex cripple vs physically challenged
            \ex intellectual disability vs special needs
            \ex blacks, coloureds vs African-American
            \ex hobo vs homeless
            \ex fat vs plus-sized
            \ex chairman vs chairperson
        \end{exe}
        
        \exmp{Language and Gender}{The influence of standard languages can reveal attitudes and how gendered language is being decreasingly non-standard, so language elements that contained, say, misogyny, would slowly shift from standard to non-standard in favour of a more gender-neutral term.}
\end{document}