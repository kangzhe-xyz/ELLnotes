\documentclass[../main.tex]{subfiles}

\begin{document}
    \section{English as a World Language}
    
    \subsection{Classifying English: the Three-Circle Model}
        A prominent linguist Braj Kachru proposed a the \textbf{Three-Circle Model}, or the Kachru's Circles of English, which classifies the English language as used by the global population: Inner, Outer, Expanding Circles. In general, 

        \edfn{Inner Circle}{Countries in the Inner Circle are countries that speak English as a first (native) language (ENL/L1). The Inner Circle can be seen as \textbf{norm-providing}---they provide the language varieties for other countries to take standards from. For instance, SSE is taken from BrE.}
        \edfn{Outer Circle}{Countries in the Outer Circle are countries that speak English as a second language (ESL/L2). The Outer Circle can be seen as \textbf{norm-developing}---they are developing their own norms for use of the English language within their own country.}
        \edfn{Expanding Circle}{Countries in the Expanding Circle are countries that speak English as a foreign language (EFL). These countries do not use English extensively on a daily basis, and for education, English is not learnt throughout their entire curriculum. The Expanding Circle can be seem as \textbf{norm-dependent}---they rely on the usage of English based on the countries in the Inner Circle rather than forming their own standards of using the English language.}

        \subsubsection{Limitations of the Three-Circle Model}
            According to Kachru, membership in the Inner Circle is permanent---countries cannot move into and out of the Inner Circle. However, this is rather contradictory as we observe the language usage patterns of Singapore, where English is turning into a home language for most households, and might be slowly moving from the Outer Circle to the Inner Circle. This shows that one problem with this model is that countries are not static in their circles as language usage patterns vary over time. \par
            Another problem with this model is that countries are not necessarily homogenous enough to be defined. For instance, India is a very large country with many regions with varying usage patterns of language. It is not representative to simply classify India as an Outer Circle country just because English is used for legislative, judiciary, and institutions of the country. \par
            An alternative to the model is to classify the population instead of the country as a whole as ENL, ESL, and EFL speakers. 

    \subsection{The Notion of a World (Global) Language}

        \edfn{World Language}{A world language satisfies the following criteria: \begin{itemize}
            \item Refers to a language that is learned and spoken internationally, and is characterised not only by the number of its native and second language speakers, but also by its geographical distribution, and its use in international organisations in diplomatic relations.
            \item Acts as a lingua franca, a common language that enables people from diverse backgrounds and ethnicities to communicate on a more or less equitable basis.
            \item Historically, the essential factor for the establishment of a global language is that it is spoken by those who wield power.
        \end{itemize}}
        \exmp{World Language: Latin}{Latin was the lingua franca of its time, although it was only ever a minority language within the Roman Empire as a while. Crucially, though, it was the language of the powerful leaders and administrators and of the Roman military---and, later, of the ecclesiastical power of the Roman Catholic Church---and this is what drove its rise to (arguably) global language status. Thus, language can be said to have no independent existence of its own, and a particular language only dominates when its speakers dominate (and, by extension, fails when the people who speak it fail).}
        The influence of any language is a combination of three main things: \begin{enumerate}
            \item the number of countries using it as their first language or mother-tongue,
            \item the number of countries adopting it as their official language, and
            \item the number of countries teaching it as their foreign language of choice in schools.
        \end{enumerate}
        The intrinsic structural qualities of a language, the size of its vocabulary, the quality of its literature throughout history, and its association with great cultures or religions, are all important factors in the popularity of any language. But, at base, history shows us that a language becomes a global language mainly due to the political power of its native speakers, and the economic power with which it is able to maintain and expand its position.
        
    \subsection{Necessity of a World Language}
        In a world of modern communications, globalised trade, and easy international travel, a single lingua franca has never been more important. \par 
        The advent since 1945 of large international bodies such as the United Nations and its various offshoots---the UN now has over 50 different agencies and programs from the World Bank, World Health Organisation, and UNICEF, to more obscure arms like the Universal Postal Union. As well as collective organisations such as the Commonwealth and the European Union, the pressure to establish a worldwide lingua franca has never been greater. As just one example of why a lingua franca is useful, consider that up to one-third of the administration costs of the European Community is taken up by translations into the various member languages. \par 
        To address this need for a world language, some have seen or planned or constructed language as a solution. In the short period between 1880 and 1907, no less than 53 such ``universal artificial languages'' were developed. Today the best known is Esperanto, a deliberately simplified language, with just 16 grammatical rules, no definite articles, no irregular endings, and no illogical spellings. \par
        Many of these universal languages were specifically developed with the view in mind that a single world language would automatically lead to world peace and unity. Setting aside for now the fact that such languages have never gained much traction, it has to be said this assumption is not necessarily well-founded. For instance, historically, many wars have broken out within communities of the same language (eg. the British and American Civil Wars, the Spanish Civil War, Vietnam, former Yugoslavia, etc.) and, on the other hand, the citizens of some countries with multiple languages manage to coexist, on the whole, quite peaceably. \par 
        People want to diversify and distinguish themselves as identity is very important---forcing one and only one world language on everyone could erode one's personal identity, and, by extension, erode cultures as other languages that can represent concepts that are unique their cultures that English cannot express as well. '

    \subsection{Consequences of a World Language}
        \subsubsection{Language Death}
            There is a risk that the increased adoption of a global language may lead to the weakening and eventually the disappearance of some minority languages (and, ultimately, it is feared, all other languages). It is estimated that up to 80\% of the world's 6000 or so living languages may die out within the next century, and some commentators believe that a too-dominant global language may be a major contributing factor in this trend. \par
            However, it seems likely that this is really a direct threat in areas where the global language is the natural first language (eg. North America, Australia, Celtic parts of Britain, etc). Conversely, there is also some evidence that the very threat of subjugation by a domination language can actually galvanise and strengthen movements to support and protect minority languages. 

        \subsubsection{Unfairness}
            There is a concern that natural speakers of the global language may be at an unfair advantage over those who are operating in their second, or even third language. The insistence on one language to the exclusion of others may also be seen as a threat to freedom of speech and to the ideals of multiculturalism. There are some concepts that are unique to a certain culture that can be expressed in one language not be as fairly represented in another global language because of differences in culture. Likewise, \par 
            Another pitfall is \textbf{linguistic complacency} on the part of natural speakers of global language---a laziness and arrogance resulting from the lack of motivation to learn other languages. Arguably, this can already be observed in many Britons and Americans.

    \subsection{English as a Candidate for a World Language}
        English is the nearest thing there has ever been to a world language. Its worldwide reach is much greater than anything achieved historically by Latin or French, and there has never been a language as widely spoken as English. \par 
        Many would reasonably claim that, in the fields of business, academia, science, computing, education, transportation, politics, and entertainment, English is already established as the de facto lingua franca. \par 
        The UN, the nearest thing we have, or have ever had, to a global community, currently uses five official languages: English, Spanish, French, Russian, Chinese. Furthermore, an estimated 85\% of international organisations have English as at least one of their official languages (French comes next with less than 50\%). Even more starkly, though, about one-third of international organisations (including OPEC, EFTA, and ASEAN) use English only, and this figure rises to almost 90\% among Asian international organisations.  \par 
        A world language arises mainly due to the political and economic power of its native speakers. It was British imperial and industrial power that sent English around the globe between the 17th and 20th Century. The legacy of British imperialism has left many counties with the language thoroughly institutionalised in their courts, parliament, civil service, schools, and higher education establishments. In other countries, English provides a neutral means of communication between ethnic groups. \par
        But it has been largely American economic and cultural supremacy---in music, film and television, business and finance, computing, information technology, and the internet, even drugs and pornography---that has consolidated the position of the English language and continues to maintain it today. \par 
        American dominance and influence worldwide makes English crucially important for developing international markets, especially in the areas of tourism and advertising, and mastery of English also provides access to scientific, technological, and academic resources which would otherwise be denied from developing countries. 

        \subsubsection{Appropriacy of English as a World Language}
        The richness and depth of English's vocabulary sets it apart from other languages. The 1989 revised ``Oxford English Dictionary'' lists \num{615000} words in 20 volumes, officially the world's largest dictionary. If technical and scientific words were to be included, the total would rise to well over a million. y some estimates, the English lexicon is currently increasing by over 8500 words a year, although estimates put this as high as \num{15000} to \num{20000}. \par
        It is estimated that about \num{200000} English words are in common use, as compared to \num{184000} in German, and mere \num{100000} in French. The availability of large numbers of synonyms allows shades of distinction that are just not available to non-English speakers and, although other languages have books of synonyms, none have anything on quite the scale of ``Roget's  Thesaurus''. Add to this the wealth of English idioms and phrases, and the available material with which to express meaning is truly prodigious, whether the intention is poetry, business, or just everyday conversation. \par 
        English is also a very flexible language---in the respect of word order and the ability to phrase sentences as active or passive. Another is in the ability to use the same word as both a noun and a verb. New words can easily be created by the addition of prefixes or suffixes, or by compounding or fusing existing words together. \par
        Its grammar is generally simpler than most languages. It dispenses completely with noun genders and often dispenses with the article completely. The distinction between familiar and formal addresses were abandoned centuries ago. Case forms for nouns are almost non-existent (with the exception of some personal pronouns like I/me/mine) as compared to Finnish, for example, which ahs fifteen forms for every noun, or Russian which has 12. In German, each verb has 16 different forms (Latin has a possible 120), while English only retains five at most (eg. ride, rides, rode, riding, riden) and often only requires three  (eg. hit, hits, hitting). \par
        Some would also claim that it is also a relatively simple language in terms of spelling and pronunciation, although this claim is perhaps more contentious. While it does not require the mastery of subtle tonal variations of Cantonese, nor the bewildering consonant clusters of Welsh or Gaelic, it does have more than its fair share of apparently random spellings, silent letters, and phonetic inconsistencies (consider, for example, the pronunciation of ``ou'' in thou, though, thought, through, thorough, tough, thorough, plough, and hiccough). \par
        There are somewhere between 44 and 52 unique sounds used in English pronunciation (depending on the authority consulted), although equally divided between vowel sounds and consonants, as compared to 26 in Italian, for example, or just 13 in Hawaiian. This includes some sounds which are notoriously difficult for non-native speakers to pronounce (such as ``th'', which comes in two varieties, as in \textit{thought} and \textit{though}), and some sounds which have a huge variety of possible spellings. In its defence, though, its consonants at least are fairly regular in pronunciation, and it is blessedly free of the accents and diacritical marks which festoon many other languages. Furthermore, borrowings of foreign words tend to preserve the original spelling (rather than attempting to spell them phonetically). It has been estimated that 84\% of English spellings conform to general patterns or rules, while only 3\% are completely unpredictable\footnote{3\% of a very large vocabulary is, however, quite large, and this includes such extraordinary examples such as \textit{colonel}, \textit{ache}, \textit{eight}.}. Arguably, some of the inconsistencies do help to distinguish between homophones like \textit{fissure} and \textit{fisher}; \textit{seas} and \textit{seize}, \textit{air} and \textit{heir}; \textit{aloud} and \textit{allowed}, etc. \par
        The `cosmopolitan' character of English (from its adoption of thousands of words from other languages with which it came into contact) gives it a feeling of familiarity and welcoming compared to many other languages (such as French, for example, which ahs tried its best to keep out other languages). Despite a tendency towards jargon, English is generally reasonably concise compared to many languages, as can be seen in the length of translations (a notable exception is Hebrew translations, which are usually shorter than their English equivalents by up to a third). It is also less prone to misunderstandings due to cultural subtleties than, say, Japanese, which is almost impossible to simultaneously translate for that reason. \par
        The absence of coding for social differences (common in many other languages which distinguish between formal and informal verb forms and sometimes other more complex social distinctions) may make English seem more democratic and remove some of the potential stress associated with language-generated social blunders. \par 
        The extent and quality of English literature throughout history marks it as a language of culture and class. As a result, it carries with it a certain legitimacy, substance, and gravitas that few other languages can match.

        \subsubsection{The Future of English}
        Although English currently appears to be in an unassailable position in the modern world, its future as a global language is not necessarily assured. In the Middle Ages, Latin seemed forever set as the language of education and culture, as did French in the 18th Century. \par 
        There are two competing drives to take into account: the pressure for international intelligibility, and the pressure to preserve national identity. It is possible that a natural balance may be achieved between the two, but it should also be recognized that the historical loyalties of British ex-colonies have been largely replaced by pragmatic utilitarian reasoning. \par 
        The very dominance of an outside language or culture can lead to a backlash or reaction against it. People do not take kindly to having a language imposed on them, whatever advantage and value that language may bring to them. As long ago as 1908, Mahatma Gandhi said, in the context of colonial India: ``To give millions a knowledge of English is to enslave them.'' \par
        Although most former British colonies retained English as an official language after independence, some (eg. Tanzania, Kenya, Malaysia) later deliberately rejected the old colonial language as a legacy of oppression and subjugation, disestablishing English as even a joint official language. Even today, there is a certain amount of resentment in some countries towards the cultural dominance of English, and particularly the USA. \par 
        As has been discussed, there is a close link between language and power. The USA, with its huge dominance in economic, technical, and cultural terms, is the driving force behind English in the world today. However, if the USA were to lose its position of economic and technical dominance, then the ``language loyalties'' of other countries may well shift to the new dominant power. Perhaps the only possible candidate for such a replacement right now would be China, but it is not that difficult to imagine circumstances in which it could happen. \par 
        A 2006 report by the British Council suggests that the number of people learning English is likely to continue to increase over the next 10--15 years, peaking at around two billion, after which a decline is predicted. Various attempts have been made to develop a simpler ``controlled'' English language suitable for international usage (eg. Basic English, Plain English, Globish, International English, Special English, Essential World English, etc.). Increasingly, the long-term future of English as a global language probably lies in the hands of Asia, and especially the huge populations of India and China. Having said that, though, there may now be a critical mass of English speakers throughout the world which may make its continued growth impossible to stop or even slow. There are no comparable historical precedents on which to base predictions, but it well may be that the emergence of English as a global language is a unique, even an irreversible, event.

\end{document}