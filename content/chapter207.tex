\documentclass[../main.tex]{subfiles}

\begin{document}
    \section{English as a World Language}
        \subsection{Classifying English: the Three-Circle Model}
        A prominent linguist Braj Kachru proposed a the \textbf{Three-Circle Model}, or the Kachru's Circles of English, which classifies the English language as used by the global population: Inner, Outer, Expanding Circles. In general, 

        \edfn{Inner Circle}{Countries in the Inner Circle are countries that speak English as a first language (EFL/L1). The Inner Circle can be seen as \textbf{norm-providing}---they provide the language varieties for other countries to take standards from. For instance, SSE is taken from BrE.}
        \edfn{Outer Circle}{Countries in the Outer Circle are countries that speak English as a second language (ESL/L2). The Outer Circle can be seen as \textbf{norm-developing}---they are developing their own norms for use of the English language within their own country.}
        \edfn{Expanding Circle}{Countries in the Expanding Circle are countries that speak English as a foreign language (EFL). These countries do not use English extensively on a daily basis, and for education, English is not learnt throughout their entire curriculum. The Expanding Circle can be seem as \textbf{norm-dependent}---they rely on the usage of English based on the countries in the Inner Circle rather than forming their own standards of using the English language.}

            \subsubsection{Limitations of the Three-Circle Model}
            According to Kachru, membership in the Inner Circle is permanent---countries cannot move into and out of the Inner Circle. However, this is rather contradictory as we observe the language usage patterns of Singapore, where English is turning into a home language for most households, and might be slowly moving from the Outer Circle to the Inner Circle. This shows that one problem with this model is that countries are not static in their circles as language usage patterns vary over time. \par
            Another problem with this model is that countries are not necessarily homogenous enough to be defined. For instance, India is a very large country with many regions with varying usage patterns of language. It is not representative to simply classify India as an Outer Circle country just because English is used for legislative, judiciary, and institutions of the country.
\end{document}