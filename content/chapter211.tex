\documentclass[../main.tex]{subfiles}

\begin{document}
    \section{Language and Ethnicity}
    \subsection{Concept of Ethnicity}
    Some ideas proposed by Edwards (1994): \begin{itemize}
        \item Allegiance to a group
        \item Regardless of size or dominance
        \item Ancestral links
        \item Some form os socialisation or cultural patterns
        \item Group boundary must persist (sustained by shared objective characteristics such as language and religion)
    \end{itemize}
    In some instances, the word `ethnic' is used to describe minority groups. But just like how everyone has a accent, everyone has an ascribed ethnicity. The difference lies in how `marked' your origins are claimed to be. \par 
    Dominant groups within a particular societal context won't see themselves as `ethnic' or with `accent'.
    \exmp{Ethnic}{``I'm going to eat some ethnic food tonight'', the usage of ``ethnic'' (eg in America, where it is used by a white person) is used to refer to minorities (or the object that is not white). It is possibly a manifestation of the attitude that they are the majority, and the minority group is marked.}

    \subsubsection{The Term `Ethnicity' Used in Context}
    In some contexts, the word `ethnic' has undergone some form of semantic pejoration, it embodies the idea of `us' vs `them'. One's membership into a particular ethnic group would also depend on the ratification of others. \par
    Edwards suggests two ways of defining ethnicity: objective and subjective.
    Objective is defined by using linguistic, racial, geographical, religious, and ancestral criteria. Ethnicity may be verified with these criteria (although not very easily). Subjective definition is the subjective belief in a common descent, and is based around shared values and heritage.

    \subsection{Ethnicity, the Nation State, and Multilingualism}
    A relationship between nation, language, and ethnicity. This relationship may be straightforward and stable, or complex and multi-faceted.
    
    \subsubsection{Multilingualism}
    Multilingualism can exist officially or unofficially (Canada v India)\par
    Not always viewed positively; in America, there is a perception that immigrants of Hispanic origins are threatening the dominance of English. This is not the case as most immigrants were noted to have learnt English in order to have access to cultural capita. \par
    
    \subsubsection{Multienglishes}
    Aside from the technical elements which it possesses, a language is also as much about the politics and power which it can rally. William Labov (1969) argued that varieties such as AAVE are not substandard forms but come with their own system, grammar, and logic. \par
    Labov analysed the speech of two individuals, one spoke Standard English while the other used AAVE. The AAVE speaker was more highly skilled and was able to present ideas with better abstract reasoning than the one who used StdE. Non-standard varieties are typically associated with the ethnic minority: the dominant group uses the dominant variety. Attitudes towards such minority varieties are in general viewed negatively, and also are fixated that the language is `incorrect'. It becomes problematic because the minority varieties will have their own set of rules, but these rules become construed to be inferior and borne out of mistakes, followed by a slippery slope argument that these speakers are unable to master fluency.

    \subsection{Ethnicity and Racism}
    Ethnicity may be associated with the minority group. Distinctions between the in-group and the out-group may be made based on ethnicity. Ethnicity may be something that is claimed or attributed. \par
    The power differences between groups of various ethnicities may contribute to one group being less valued than the other (despite pains taken to establish an authentic ethnic identity). \par
    Teun van Dijk (2004) identified racist discourse as `a form of discriminatory social practice that manifests itself in text, talk, and communication.' One of the most obvious ways racist discourse is manifested is through the use of pejorative words. Other ways may include shutting out certain groups in a conversation through topic setting, not allowing airtime, responding negatively, or not using a mutually intelligible language. \par
    In discussing the `self vs other' mentality in terms of drawing out ethnic boundaries, van Dijk (2004) also identified these three topic classes when communicating with the `other': \begin{enumerate}
        \item difference,
        \item deviance, and
        \item threat.
    \end{enumerate}
    
    \textbf{Denial} is also something which is commonly observed in the case of racist discourse. For instance, someone would say ``I'm not racist, but...''. 
    \exmp{Racist Discourse}{\textit{``Our tradition of fairness and tolerance are being exploited by every terrorist, crook, screwball and scrounger who wants a free ride at our expense... Then there are the criminals who sneak in as political refugees or as family members visiting a distant relative.''} \\ taken from Daily Mail, 1985-11-28}
    \exmp{COVID-19 in Singapore}{Delibrate separation of demographics when reporting new cases: \(x\) Singaporeans/PR, \(y\) Work Pass Holders, \(z\) in Dormitories. To help manage possible anxiety, the Government decided that maybe if we separate the numbers it will alleviate. }
\end{document}