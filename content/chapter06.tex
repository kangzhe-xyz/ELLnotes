\documentclass[../main.tex]{subfiles}

\begin{document}
\section{Discourse Analysis}

    \subsection{Texture}
    A text is a passage, spoken or written, that constructs a unified whole. 
    
    A text is a unit of language in use.
    
    A text is not made up of sentences---it is realised by sentences.
    
    \edfn{Texture}{Texture is the quality that makes a set of words of sentences a text rather than a random collection of linguistic terms.}
    
    The texture of the text is determined by relationships between words and sentences.

    \subsection{Cohesion}
    \edfn{Cohesion}{Cohesion is the sense of semantic ``connectedness'' a text has.}

    To determine cohesion, we can look at \textbf{lexical cohesion} (lexical chains, repetition) or \textbf{grammatical cohesion} (conjunctions, referencing).

    \subsection{Grammatical Cohesion}
        \subsubsection{Referencing}
        \edfn{Exophoric Referencing}{Exophoric referencing (deictic referencing) is when the reference to the referent is not in the text. This might suggest that TP and TR have common knowledge of the referent.}
        \edfn{Endophoric Referencing}{Endophoric referencing is when the reference to the referent is in the text.}
        Endophoric referencing exists in two types:\begin{itemize}
            \item anaphoric: backwards (referencing something mentioned); the referent coming before the anaphoric reference is known as the antecedent.
            \item cataphoric: forwards (referencing something to be mentioned)
        \end{itemize}

        \subsubsection{Conjunctions}
        \edfn{Conjunction}{A conjunction is a device that joins two or more clauses together.}

    \subsection{Theme and Rheme}
        In every clause, there is a theme and a rheme. 
        \edfn{Theme and Rheme}{A \textbf{theme} [Th] is the point of departure of a clause; a \textbf{rheme} [Rh] is what supplements the theme.}

        A theme is a clause-initial element. It provides the context of the message carried within the clause. It also serves as the ``departure point'' of the sentence.
        
        In a main clause, there should be \textbf{at least one} topical theme. 

        A theme \textbf{need not} be the subject. Themes do not have to be a noun phrase (NP).
        \exmp{Theme and Rheme}{\begin{exe}
            \ex \undertext{I}{Th} \undertext{am a boy}{Rh}.
            \ex \undertext{This}{Th} \undertext{is a book}{Rh}.
            \ex \undertext{A book}{Th} \undertext{this is}{Rh}.
        \end{exe}}

        There are three types of themes:\begin{itemize}
            \item Textual theme: continuatives, conjunctions, conjuctive adverbs, relative pronouns
            \item Interpersonal theme: vocatives, modal auxillary, finite operators, ``wh-'' question words
            \item Topical theme: realised by participant (NP), processes (VP), and the circumstance in which the event happened (PP) or (AdvP)
        \end{itemize}
        Employ SVOCA to find the topical theme

        \edfn{Marked Theme}{A (topical) theme is said to be \textbf{marked} when it is not the subject of the sentence.}
        \exmp{Marked Theme}{\begin{exe}
            \ex \undertext{You}{Th} \undertext{must have patience my young Padawan}{Rh}.
            \ex \undertext{Patience}{Th} \undertext{you must have my young Padawan}{Rh}.
        \end{exe}}

    \subsection{Thematic Progression}
    Thematic progression looks at how themes are \textbf{developed}. The idea of development comes from building upon previous sentences/linguistic units. Thematic progression might give us an insight to the plannedness of a text---the more well-structured the thematic progression is, the higher the degree of planning.

        \subsubsection{Progression Types}
        \textbf{Linear}

\end{document}