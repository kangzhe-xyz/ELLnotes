\documentclass[../main.tex]{subfiles}

\begin{document}
\chapter{Sentence Structure II}
\section{Sentence Complexity}
Sentences can be analysed according to the complexity of the structure. There are 5 main types of sentences.
\edfn{Simple Sentence}{A simple sentence consists of an independent clause.}
\edfn{Compound Sentence}{A compound sentence contains two clauses that are in a \textbf{coordinating relationship}---they are joined by a coordinating conjunction (eg. and).}
\edfn{Complex Sentence}{A complex sentence contains two clauses that are in a \textbf{subordinating/hierarchical relationship}---they are joined by a subordinating conjunction (eg. or).}
\edfn{Compound-complex Sentence}{A compound-complex sentence has the features of both a compound sentence and a complex sentence.}
\edfn{Minor Sentence}{A minor sentence has something missing from SVOCA.}
\exmp{Sentence structure examples}{In the order introduced above, \begin{exe}
    \ex I went to the barber yesterday.
    \ex I went to class \textit{but} I did not see anyone there.
    \ex I went to the fridge \textit{because} I was hungry.
    \ex I went to the fridge \textit{because} I was hungry \textit{but} I did not see anyone there.
    \ex Go.
\end{exe}}
\end{document}
