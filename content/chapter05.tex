\documentclass[../main.tex]{subfiles}

\begin{document}
	\chapter{Syntax III: Sentence Mood}
	\begin{preamb}
		If we assume that language is an exchange of information, then the manner of this exchange tells us the text producer's attitude towards the event and the relationship he has with the text recipient.
	\end{preamb}
	
	\section{Sentence Types}
		\edfn{Declaratives}{A declarative sentence said by a TP is something that the TP deems to be true. \par Syntax: V Subj (Obj)}
		\exmp{Declaratives}{\begin{exe}
			\ex He is sleeping.
			\ex He is not sleeping.
			\ex He slept.
			\ex He sleeps.
		\end{exe}}
		Declarative sentences state truth, and when a declarative statement is confident, we say it has \textit{high epistemic modality}.

		\edfn{Imperatives}{An imperative sentence said by a TP is a command the TP expects the TR (or the appropriate party) to carry out. \par Syntax: V (Obj)}
		Imperatives usually have implied second person singular pronouns ``you'', but it is not unusual to see them explicitly stated.
		\exmp{Imperatives}{\begin{exe}
			\ex Think different.
			\ex Keep calm and carry on.
			\ex \begin{exe}
				\ex Close the door.
				\ex You close the door.
			\end{exe}
		\end{exe}}
		Imperatives can be more than commands; they can also be requests, granting/denying permission, making offers, apologies etc. 

		\edfn{Interrogatives}{An interrogative sentence is a question posed by a TP. \par Syntax: Wh- question words/verb-to-be/do first.}
		Interrogatives are usually formed through the process of \textit{inversion}.
		\exmp{Interrogatives}{
			\begin{exe}
				\ex He kicked the boy.
				\ex Did he kick the boy?
				\ex You are coming to the party.
				\ex Are you coming to the party?
			\end{exe}
		}

		\edfn{Exclamatives}{An exclamative sentence expresses strong emotion.}
		An exclamative need not have an exclamation mark [!]. Its primary purpose is to display emotion or feelings.
		\exmp{Exclamatives}{
			\begin{exe}
				\ex I did it!
				\ex What a beautiful country!
				\ex Excellent work!
			\end{exe}
		}

	\section{Sentence Mood Analysis}
	Sentence mood analysis is a good tool to gain some insight into relationships between the TP and TR, and the attitudes the participants have towards any events. The effectiveness of the sentence depends on the audience, purpose, and context.
	
\end{document}
