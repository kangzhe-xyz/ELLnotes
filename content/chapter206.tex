\documentclass[../main.tex]{subfiles}

\begin{document}
    \section{Colloquial and Standard Singapore English}
    
    \exmp{Singlish particle `\textit{lah}'}{
    While many may find the use of \textit{lah} obvious or intuitive, some have identified subtler layers of meaning behind Singlish's most well known particle. In her dissertation, Vivien Ler (2005) proposed a few categories on how \textit{lah} is used in Singlish. \par

    \textbf{Obviousness} \par 
    Context: Aminah (A) tells Irene (I) that a mutual friend of their has got a job at the Institute of Education. Irene asks how does one get the job. \par
    I: How do you get to do that? \\
    A: Apply \textit{lah}. \par

    \textbf{Friendliness} \par
    Utterance: Come with us \textit{lah}. \par

    \textbf{Impatience} \par 
    Context: A mother (M) and her daughter (D) had a disagreement on who is to buy Mandarin Oranges. \par

    M: Then after that it's the Chinese New Year special \textit{lah}. \\
    D: So? \\
    M: Ya \textit{lah}, then during that period we can go what? \\
    D: Cannot \textit{lah}. Aiyah, when I wash my hair, I don't want to go out. Dirty my hair \textit{lah}. \\
    M: You bring one of them \textit{lah}. \par

    \textbf{Emphasis} \par
    A: Do you want to go? \\
    B: I'm not going \textit{lah}. \par
    }
\end{document}